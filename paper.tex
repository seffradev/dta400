\documentclass[conference]{IEEEtran}
\IEEEoverridecommandlockouts
\usepackage{graphicx}
\usepackage{svg}
\usepackage{amsmath,amssymb,amsfonts}
\usepackage[natbib=true, style=numeric, sorting=none]{biblatex}

\addbibresource{references.bib}

\begin{document}

\title{Satisfaction When Waiting in a Queue}

\author{
    Hampus Avekvist \\
    \IEEEauthorblockA{
        \textit{Department of Engineering Sciences} \\
        \textit{University West}\\
        Trollhättan, Sweden \\
        hampus.avekvist@hey.com
    }
}

\maketitle

\begin{abstract}
    the whole concept is abstract, bruv, and then comes maths
    and suddenly it's a whole new level of shenanigans.
\end{abstract}

\begin{IEEEkeywords}
    satisfaction modeling
\end{IEEEkeywords}

\section{Introduction}

\section{Method}

\subsection{The model}

The model is a satisfaction-oriented $M/M/c$ \cite{Citation maybe?}
(see Fig.~\ref{fig:queue-diagram}) system with $c$ services, where
actors start with a uniformly distributed level of satisfaction
in the range $[2, 5]$. There is a secondary range, $[0, 5]$, for
the current satisfaction of an actor used for measurements in the
simulation. The ranges are arbitrarily chosen. $0$ is the lowest
level of satisfaction and will prompt an actor to leave the system.
While the actor is in the queue, the satisfaction will decrement,
alongside if they face any difficulties using a service. 

Actors join the queue at a uniform interval of $[0, 4]$ minutes
and the service rate is, on average $placeholder$ minutes. If an
actor faces difficulties using the service, an arbitrary extra 30
seconds is added to the simulation time and the actor retries. The
actor retries until either the level of satisfaction drops to $0$
or until they have finished using the service.

\begin{figure}[!b]
    \centerline{
        \includesvg[width=0.25\textwidth]{figures/queue-model.svg}
    }
    \caption{
       A visual representation of what the queue can look like.
       Circles are actors, while the boxes at the top are service
       desks.
    }
    \label{fig:queue-diagram}
\end{figure}

\subsubsection{Limitations}

The following limitations provide possible ideas for further work.

\begin{itemize}
    \item The model is limited to an unoccupied queue
        \cite{MakingSomeoneWait}.
    \item The model doesn't allow people to know how long they
        will be standing in line \cite{MakingSomeoneWait}.
    \item The model assumes no cultural bias
        \cite{MakingSomeoneWait}.
    \item The model is a serpentine and not a parallel queue
        \cite{MakingSomeoneWait}.
    \item Not based on real data. Data from real experiments with
        similar scenarios are required.
    \item The model only accounts for satisfaction. A person
        could e.g. be more or less patient which may affect total
        satisfaction but won't be factored in this model.
    \item The queue requires active participation to remain in
        place and is not a ``sign-up and be notified'' type of
        queue that would allow simultaneous activity.
    \item The queue is ``perpetual`` and people can come and go
        as they like. A realistic situation would be a queue tied
        to working hours that could affect customer satisfaction
        from no longer accepting requests past a certain time.
\end{itemize}

\section{Results}

\section{Discussion}

\section{Conclusion}

\printbibliography

\end{document}
